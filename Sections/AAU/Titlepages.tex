\pdfbookmark[0]{English title page}{label:titlepage_en}
\aautitlepage{%
  \englishprojectinfo{
    ROB3-Project %title
  }{%
     %theme
     Robotics
  }{%
    1$^{st}$ of September to the 18$^{th}$ of December 2018 %project period
  }{%
    362 % project group
  }{%
    %list of group members
    Peter Birk Torrild\\ 
    Casper Lindholm Fl\o e Mikkelsen\\
    Valdemar Qvist\\
    Mark Blankensteiner
  }{%
    %list of supervisors
    John Hansen
  }{%
    1 % number of printed copies
  }{%
    \today % date of completion
  }%
}{%department and address
  \textbf{Robotics\\ Department of Electronic Systems}\\
  Fredrik Bajers Vej 7B\\
  DK-9220 Aalborg\\
  \href{http://es.aau.dk/}{http://es.aau.dk/}
}{% the abstrac
The Goal of this project was to investigate the viability in using a manipulator such as the CrustCrawler to assist users with shoulder disarticulation in interacting with their surroundings. 
By investigating several potential options to control such a manipulator, and reflecting on the easiest ways for the user to control the manipulator, it was chosen to utilise the users own biological signals as the control method.\\
The finished system was tested and met most requirements set by the project group, although further refinement could yield better results .\\
The system have the capability of assisting a wide range of potential users, as the only criteria for controlling the system would be two independent muscles and the mobility to shake a body part, thereby choosing the motor to adjust. As such the problem formulation of, \textit{\textbf{"How can a person with a shoulder dis-articulation through the use of biological signals control a robotic manipulator."}}
is answered.
}


