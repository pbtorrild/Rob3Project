In order to use it in 3D space, with a manipulator with rotational joints, the equation can be expanded to:
\begin{equation}\label{jacobian3d}{
\begin{bmatrix}
\dot{x} \\
\dot{y} \\
\dot{z} \\
\omega_x \\
\omega_y \\
\omega_z 
\end{bmatrix} 
} = 
J \cdot 
\begin{bmatrix}
\dot{\theta_1} \\
\dot{\theta_2} \\
\dot{\theta_3} \\
\end{bmatrix} 
\end{equation}

To calculate the Jacobian(J), a homogeneous transformation matrix(HTM) is to be defined, to do this the denavit-hartenberg parameters will be established using the frames shown on figure \ref{CCframesDH}.


\begin{figure}[H]
    \centering
    \begin{tikzpicture}[x=0.75pt,y=0.75pt,yscale=-1,xscale=1]
%uncomment if require: \path (0,496); %set diagram left start at 0, and has height of 496

%Shape: Can [id:dp5296973039358415] 
\draw   (187,400) -- (187,452) .. controls (187,456.97) and (173.57,461) .. (157,461) .. controls (140.43,461) and (127,456.97) .. (127,452) -- (127,400)(187,400) .. controls (187,404.97) and (173.57,409) .. (157,409) .. controls (140.43,409) and (127,404.97) .. (127,400) .. controls (127,395.03) and (140.43,391) .. (157,391) .. controls (173.57,391) and (187,395.03) .. (187,400) -- cycle ;
%Straight Lines [id:da28195955926211136] 
\draw    (155.5,329) -- (156,399) ;


%Straight Lines [id:da5703997191030459] 
\draw    (156.5,197) -- (155.5,279) ;


%Straight Lines [id:da34725458649816576] 
\draw    (156.5,92) -- (156.5,147) ;


%Shape: Half Frame [id:dp6174190186388537] 
\draw   (156.64,92.48) -- (108.52,41.64) -- (138.21,42.46) -- (157.46,62.79) -- (177.79,43.54) -- (207.48,44.36) -- cycle ;
%Straight Lines [id:da34284815490068854] 
\draw    (263.5,310) -- (263.5,396) ;
\draw [shift={(263.5,396)}, rotate = 270] [color={rgb, 255:red, 0; green, 0; blue, 0 }  ][line width=0.75]    (0,5.59) -- (0,-5.59)   ;
\draw [shift={(263.5,310)}, rotate = 270] [color={rgb, 255:red, 0; green, 0; blue, 0 }  ][line width=0.75]    (0,5.59) -- (0,-5.59)   ;
%Straight Lines [id:da7031226841516269] 
\draw    (262.5,171) -- (263.5,310) ;
\draw [shift={(263.5,310)}, rotate = 269.59000000000003] [color={rgb, 255:red, 0; green, 0; blue, 0 }  ][line width=0.75]    (0,5.59) -- (0,-5.59)   ;
\draw [shift={(262.5,171)}, rotate = 269.59000000000003] [color={rgb, 255:red, 0; green, 0; blue, 0 }  ][line width=0.75]    (0,5.59) -- (0,-5.59)   ;
%Straight Lines [id:da033726263026289116] 
\draw    (262.5,47) -- (262.5,171) ;
\draw [shift={(262.5,171)}, rotate = 270] [color={rgb, 255:red, 0; green, 0; blue, 0 }  ][line width=0.75]    (0,5.59) -- (0,-5.59)   ;
\draw [shift={(262.5,47)}, rotate = 270] [color={rgb, 255:red, 0; green, 0; blue, 0 }  ][line width=0.75]    (0,5.59) -- (0,-5.59)   ;
%Shape: Axis 2D [id:dp2927370804262843] 
\draw [color={rgb, 255:red, 255; green, 2; blue, 2 }  ,draw opacity=1 ] (32,305.1) -- (72.5,305.1)(36.05,270) -- (36.05,309) (65.5,300.1) -- (72.5,305.1) -- (65.5,310.1) (31.05,277) -- (36.05,270) -- (41.05,277)  ;
%Shape: Axis 2D [id:dp640678901747024] 
\draw [color={rgb, 255:red, 255; green, 2; blue, 2 }  ,draw opacity=1 ] (32,457.1) -- (72.5,457.1)(36.05,422) -- (36.05,461) (65.5,452.1) -- (72.5,457.1) -- (65.5,462.1) (31.05,429) -- (36.05,422) -- (41.05,429)  ;
%Flowchart: Or [id:dp5334614926566488] 
\draw  [color={rgb, 255:red, 255; green, 0; blue, 0 }  ,draw opacity=1 ] (374,151) .. controls (374,147.13) and (377.47,144) .. (381.75,144) .. controls (386.03,144) and (389.5,147.13) .. (389.5,151) .. controls (389.5,154.87) and (386.03,158) .. (381.75,158) .. controls (377.47,158) and (374,154.87) .. (374,151) -- cycle ; \draw  [color={rgb, 255:red, 255; green, 0; blue, 0 }  ,draw opacity=1 ] (374,151) -- (389.5,151) ; \draw  [color={rgb, 255:red, 255; green, 0; blue, 0 }  ,draw opacity=1 ] (381.75,144) -- (381.75,158) ;
%Flowchart: Or [id:dp9149134147886435] 
\draw  [color={rgb, 255:red, 255; green, 0; blue, 0 }  ,draw opacity=1 ] (27.85,305.7) .. controls (27.85,301.83) and (31.32,298.7) .. (35.6,298.7) .. controls (39.88,298.7) and (43.35,301.83) .. (43.35,305.7) .. controls (43.35,309.57) and (39.88,312.7) .. (35.6,312.7) .. controls (31.32,312.7) and (27.85,309.57) .. (27.85,305.7) -- cycle ; \draw  [color={rgb, 255:red, 255; green, 0; blue, 0 }  ,draw opacity=1 ] (27.85,305.7) -- (43.35,305.7) ; \draw  [color={rgb, 255:red, 255; green, 0; blue, 0 }  ,draw opacity=1 ] (35.6,298.7) -- (35.6,312.7) ;
%Shape: Axis 2D [id:dp2593810984265956] 
\draw [color={rgb, 255:red, 255; green, 2; blue, 2 }  ,draw opacity=1 ] (32,173.1) -- (72.5,173.1)(36.05,138) -- (36.05,177) (65.5,168.1) -- (72.5,173.1) -- (65.5,178.1) (31.05,145) -- (36.05,138) -- (41.05,145)  ;
%Flowchart: Or [id:dp8662160020179077] 
\draw  [color={rgb, 255:red, 255; green, 0; blue, 0 }  ,draw opacity=1 ] (27.85,173.7) .. controls (27.85,169.83) and (31.32,166.7) .. (35.6,166.7) .. controls (39.88,166.7) and (43.35,169.83) .. (43.35,173.7) .. controls (43.35,177.57) and (39.88,180.7) .. (35.6,180.7) .. controls (31.32,180.7) and (27.85,177.57) .. (27.85,173.7) -- cycle ; \draw  [color={rgb, 255:red, 255; green, 0; blue, 0 }  ,draw opacity=1 ] (27.85,173.7) -- (43.35,173.7) ; \draw  [color={rgb, 255:red, 255; green, 0; blue, 0 }  ,draw opacity=1 ] (35.6,166.7) -- (35.6,180.7) ;
%Shape: Axis 2D [id:dp7974568085279086] 
\draw [color={rgb, 255:red, 255; green, 2; blue, 2 }  ,draw opacity=1 ] (32,61.1) -- (72.5,61.1)(36.05,26) -- (36.05,65) (65.5,56.1) -- (72.5,61.1) -- (65.5,66.1) (31.05,33) -- (36.05,26) -- (41.05,33)  ;
%Flowchart: Or [id:dp7695371354385845] 
\draw  [color={rgb, 255:red, 255; green, 0; blue, 0 }  ,draw opacity=1 ] (27.85,61.7) .. controls (27.85,57.83) and (31.32,54.7) .. (35.6,54.7) .. controls (39.88,54.7) and (43.35,57.83) .. (43.35,61.7) .. controls (43.35,65.57) and (39.88,68.7) .. (35.6,68.7) .. controls (31.32,68.7) and (27.85,65.57) .. (27.85,61.7) -- cycle ; \draw  [color={rgb, 255:red, 255; green, 0; blue, 0 }  ,draw opacity=1 ] (27.85,61.7) -- (43.35,61.7) ; \draw  [color={rgb, 255:red, 255; green, 0; blue, 0 }  ,draw opacity=1 ] (35.6,54.7) -- (35.6,68.7) ;
%Flowchart: Or [id:dp8748598680447852] 
\draw  [color={rgb, 255:red, 255; green, 0; blue, 0 }  ,draw opacity=1 ] (376.27,168.52) .. controls (379,165.79) and (383.67,166.02) .. (386.7,169.05) .. controls (389.73,172.08) and (389.96,176.75) .. (387.23,179.48) .. controls (384.5,182.21) and (379.83,181.98) .. (376.8,178.95) .. controls (373.77,175.92) and (373.54,171.25) .. (376.27,168.52) -- cycle ; \draw  [color={rgb, 255:red, 255; green, 0; blue, 0 }  ,draw opacity=1 ] (376.27,168.52) -- (387.23,179.48) ; \draw  [color={rgb, 255:red, 255; green, 0; blue, 0 }  ,draw opacity=1 ] (386.7,169.05) -- (376.8,178.95) ;
%Flowchart: Or [id:dp12306352272493526] 
\draw  [color={rgb, 255:red, 255; green, 0; blue, 0 }  ,draw opacity=1 ] (31.05,453.1) .. controls (33.78,450.37) and (38.45,450.6) .. (41.48,453.63) .. controls (44.51,456.66) and (44.74,461.33) .. (42.01,464.06) .. controls (39.28,466.79) and (34.61,466.56) .. (31.58,463.53) .. controls (28.55,460.5) and (28.32,455.83) .. (31.05,453.1) -- cycle ; \draw  [color={rgb, 255:red, 255; green, 0; blue, 0 }  ,draw opacity=1 ] (31.05,453.1) -- (42.01,464.06) ; \draw  [color={rgb, 255:red, 255; green, 0; blue, 0 }  ,draw opacity=1 ] (41.48,453.63) -- (31.58,463.53) ;
%Shape: Circle [id:dp4842017742180267] 
\draw   (131.5,172) .. controls (131.5,158.19) and (142.69,147) .. (156.5,147) .. controls (170.31,147) and (181.5,158.19) .. (181.5,172) .. controls (181.5,185.81) and (170.31,197) .. (156.5,197) .. controls (142.69,197) and (131.5,185.81) .. (131.5,172) -- cycle ;
%Shape: Circle [id:dp8571189084674604] 
\draw   (130.5,304) .. controls (130.5,290.19) and (141.69,279) .. (155.5,279) .. controls (169.31,279) and (180.5,290.19) .. (180.5,304) .. controls (180.5,317.81) and (169.31,329) .. (155.5,329) .. controls (141.69,329) and (130.5,317.81) .. (130.5,304) -- cycle ;

% Text Node
\draw (134,115) node  [align=left] {Link 3};
% Text Node
\draw (131,231) node  [align=left] {Link 2};
% Text Node
\draw (134,363) node  [align=left] {Link 1};
% Text Node
\draw (101,424) node  [align=left] {Joint 1};
% Text Node
\draw (95,300) node  [align=left] {Joint 2};
% Text Node
\draw (156,34) node  [align=left] {End-Effector};
% Text Node
\draw (289,111) node  [align=left] {219mm};
% Text Node
\draw (292,239) node  [align=left] {210mm};
% Text Node
\draw (291,350) node  [align=left] {70mm};
% Text Node
\draw (405,150) node  [align=left] {Out};
% Text Node
\draw (406,170) node  [align=left] {In};
% Text Node
\draw (79,313) node   {$x_{1}$};
% Text Node
\draw (84,468) node   {$x_{0}$};
% Text Node
\draw (28,315) node   {$z_{1}$};
% Text Node
\draw (39,412) node   {$z_{0}$};
% Text Node
\draw (40,257) node   {$y_{1}$};
% Text Node
\draw (33,476) node   {$y_{0}$};
% Text Node
\draw (95,168) node  [align=left] {Joint 3};
% Text Node
\draw (79,181) node   {$x_{2}$};
% Text Node
\draw (28,183) node   {$z_{2}$};
% Text Node
\draw (37,126) node   {$y_{2}$};
% Text Node
\draw (79,69) node   {$x_{3}$};
% Text Node
\draw (28,71) node   {$z_{3}$};
% Text Node
\draw (40,13) node   {$y_{3}$};


\end{tikzpicture} \label{CCframesDH}    \caption{Diagram of frame placement on the CrustCrawler manipulator. Circles indicates the location of a joint and the measuremts on the right indicates the distance between the joints}
    \label{CCframesDH}
\end{figure}


\begin{table}[]
\centering
\begin{tabular}{|l|l|l|l|l|}
\hline
$joint\#      & a   & \alpha & d  & \theta                           \\ \hline
1            & 0   & \pi    & 70 & \theta_1                        \\ \hline
2            & 0   & \pi/2  & 0  & \theta_2 + \pi/2 \\ \hline
3            & 210 & 0                     & 0  & \theta_3                        \\ \hline
end-effector & 219 & 0                     & 0  & \theta_3     $                   \\ \hline
\end{tabular}
\caption{D-H parameters of the CrustCrawler manipulator}
\label{DH-Table2}

\end{table}

In equation \ref{jacobian3d} the jacobian (J) is a matrix with 3 columns, as the manipulator have 3 joint, and 6 rows, as this is a constant, as seen below in eq. \ref{eq:jacobiangen1}

\begin{equation}\label{eq:jacobiangen1}
J=
\begin{bmatrix}
\begin{tabular}[c]{@{}l@{}}R^0_0 $ \begin{bmatrix}\\ 0 \\ 0 \\ 1 \\ \end{bmatrix}  $\times(\(d^0_3-d^0_0)\)\end{tabular}    & \begin{tabular}[c]{@{}l@{}}\(R^0_1 \)$  \begin{bmatrix}\\ 0 \\ 0 \\ 1 \\ \end{bmatrix}  $\times(\(d^0_3-d^0_1)\)\end{tabular} & \begin{tabular}[c]{@{}l@{}}\(R^0_2\) $  \begin{bmatrix}\\ 0 \\ 0 \\ 1 \\ \end{bmatrix}$\times(\(d^0_3-d^0_2)\)\end{tabular}  \\

\begin{tabular}[c]{@{}l@{}}\(R^0_0\) $  \begin{bmatrix}\\ 0 \\ 0 \\ 1 \\ \end{bmatrix} $\end{tabular} & \begin{tabular}[c]{@{}l@{}}\(R^0_1\) $  \begin{bmatrix}\\ 0 \\ 0 \\ 1 \\ \end{bmatrix}  $ \end{tabular} & \begin{tabular}[c]{@{}l@{}}\(R^0_2\) $  \begin{bmatrix}\\ 0 \\ 0 \\ 1 \\ \end{bmatrix}  $ \end{tabular}
\end{bmatrix}
\end{equation}

Here \(R^0_n\) is the rotational matrix from the zero'th to the n'th frame and \(d^0_n\) is the translation from the zero'th to the n'th frame. these would be found when constructing the homogeneous transformation matrix(HTM) of the manipulator. The general structure is seen in \ref{genHTM} where \(H^0_n\) is the transformation from the manipulators first joint frame to the n'th frame.\\
\begin{equation}\label{genHTM}
H^0_n=
\begin{bmatrix}
 &\( R^0_n\) &  &\( d^0_n \)\\
0 & 0 & 0 & 1 
\end{bmatrix} 
\end{equation}\\
If calculated for the CrustCrawler, defining the joint angles to zero: \\

\begin{equation} \label{HTM01}
H^0_1=
\begin{matrix}
1 & 0 & 0 & 0 \\
0 & -1 & 0 & 0 \\
0 & 0 & -1 & 70 \\
0 & 0 & 0 & 1
\end{matrix}
\end{equation}

\begin{equation}\label{HTM02}
H^0_2=
\begin{matrix}
0 & 0 & -1 & 0 \\
1 & 0 & 0 & 0 \\
0 & -1 & 0 & 70 \\
0 & 0 & 0 & 1
\end{matrix}
\end{equation}

\begin{equation}\label{HTM03}
H^0_3=
\begin{matrix}
0 & 0 & -1 & 0 \\
1 & 0 & 0 & 210 \\
0 & -1 & 0 & 70 \\
0 & 0 & 0 & 1
\end{matrix}
\end{equation}


the missing values needed to calculate the jacobian in eq \ref{eq:jacobiangen1} can be found in eq \ref{HTM01}, \ref{HTM02} and \ref{HTM03} as per eq \ref{genHTM}. While \(R^0_0 \)is the identity matrix resulting in the matrix seen in eq. \ref{jacobiangen2}.


\begin{equation}\label{jacobiangen2}
J=
\begin{bmatrix}
\begin{tabular}[c]{@{}l@{}}
\begin{bmatrix}
1 & 0 & 0 \\
0 & 1 & 0 \\
0 & 0 & 1 
\end{bmatrix}  $ \begin{bmatrix}\\ 0 \\ 0 \\ 1 \\ \end{bmatrix}  $
\times \begin{bmatrix}
0 \\
210 \\
70 
\end{bmatrix}
\end{tabular}    & \begin{tabular}[c]{@{}l@{}}
\begin{bmatrix}
1 & 0 & 0 \\
0 & -1 & 0 \\
0 & 0 & -1 
\end{bmatrix} $  \begin{bmatrix}\\ 0 \\ 0 \\ 1 \\ \end{bmatrix}  $\times \begin{bmatrix}
0 \\
210 \\
0 
\end{bmatrix}\end{tabular} & \begin{tabular}[c]{@{}l@{}}
\begin{bmatrix}
0 & 0 & -1 \\
1 & 0 & 0 \\
0 & -1 & 0 
\end{bmatrix}  $  \begin{bmatrix}\\ 0 \\ 0 \\ 1 \\ \end{bmatrix}$
\times \begin{bmatrix}
0 \\
210 \\
0 
\end{bmatrix}\end{tabular}  \\

\begin{tabular}[c]{@{}l@{}}
\begin{bmatrix}
1 & 0 & 0 \\
0 & 1 & 0 \\
0 & 0 & 1 
\end{bmatrix}  
$  \begin{bmatrix}\\ 0 \\ 0 \\ 1 \\ \end{bmatrix} $\end{tabular} & \begin{tabular}[c]{@{}l@{}}
\begin{bmatrix}
1 & 0 & 0 \\
0 & -1 & 0 \\
0 & 0 & -1
\end{bmatrix}  $  \begin{bmatrix}\\ 0 \\ 0 \\ 1 \\ \end{bmatrix}  $ \end{tabular} & \begin{tabular}[c]{@{}l@{}}
\begin{bmatrix}
0 & 0 & -1 \\
1 & 0 & 0 \\
0 & -1 & 0 
\end{bmatrix}  $  \begin{bmatrix}\\ 0 \\ 0 \\ 1 \\ \end{bmatrix}  $ \end{tabular}
\end{bmatrix}
\end{equation}

finally:
\begin{equation}\label{jacobiangen3}
J=
\begin{bmatrix}
\begin{tabular}[c]{@{}l@{}}
\begin{bmatrix}
0 \\
0 \\
1 
\end{bmatrix}
\times 
\begin{bmatrix}
0 \\
210 \\
70 
\end{bmatrix}
\end{tabular}    & 
\begin{tabular}[c]{@{}l@{}}
\begin{bmatrix}
0 \\
0 \\
-1 
\end{bmatrix}
\times 
\begin{bmatrix}
0 \\
210 \\
0 
\end{bmatrix}
\end{tabular} & 
\begin{tabular}[c]{@{}l@{}}
\begin{bmatrix}
-1 \\
0 \\
0 
\end{bmatrix} 
\times 
\begin{bmatrix}
0 \\
210 \\
0 
\end{bmatrix}
\end{tabular}  \\

\begin{tabular}[c]{@{}l@{}}
\begin{bmatrix}
0 \\
0 \\
1 
\end{bmatrix}  
\end{tabular} & 
\begin{tabular}[c]{@{}l@{}}
\begin{bmatrix}
0 \\
0 \\
-1
\end{bmatrix}  
\end{tabular} & 
\begin{tabular}[c]{@{}l@{}}
\begin{bmatrix}
-1 \\
0 \\
0 
\end{bmatrix} 
\end{tabular}
\end{bmatrix}
\end{equation}

\begin{equation}
   J= 
\begin{bmatrix}
-210 & 210 & 0 \\
0 & 0 & 0 \\
0 & 0 & -210 \\
0 & 0 & -1 \\
0 & 0 & 0 \\
1 & -1 & 0
\end{bmatrix}
\end{equation}
