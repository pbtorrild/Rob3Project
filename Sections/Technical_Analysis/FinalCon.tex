\chapter{Final conclusion}\label{cp:Final-conclusion}

Controlling a manipulator and easing every day of the end-user has been the goal.\\
Persons with a high-level of amputation has been taken into consideration and it has been looked into how the struggles these persons can be overcome. By establishing a safe means of communication between the user and the prosthesis through a wireless connection and muscular contracting signals. The communication in this project is done with two standards and protocols. The wireless communication from the measuring box to the receiver was done with the Xbee protocol that utilises Bluetooth, from the receiver to the Teensy was done through the RS-232 communication protocol. Furthermore, the communication of the manipulator was done with the RS-485 and the Dynamixel servo protocol. \\
The users were able to manipulate the prosthesis, through a test-setting of the project groups choice, and thereby utilising the muscular contractions that were implemented in the control of the prosthesis.\\
To apply the needed control of muscular contractions signals further a understanding of these signals and how to handle them were analysed. The RMS was a convenient method and were already implemented in the hardware, the RMS filter gave a clear signal without must noise, this meant that the signal could be exploited, by setting some thresholds for the user, if the signal had a magnitude high then the threshold, the prosthesis will perfrom a specific task, depending on the control setup.\\
To be able to control the prosthesis accurate without letting disturbance influence the system at a minimum, the advantage of having a PID controller in each Dynamixel servo is used. In this project, only a PI controller is implemented minimise the fast response from the derivative term.\\
Analysing the different links and dynamics of the chosen prosthesis: The CrustCrawler helps to get a deeper understanding of how the motion of the chosen prosthesis works.\\
Forward and inverse kinematics is used to get a better understanding of links, joints, rotation matrices and DH-parameters. Further dynamics, such as Lagrange-equations, lagranians and the Newton-Euler method, can be used to understand the torques, forces, velocities of the vectors of the body and the inward forces that the prosthesis can utilise. Thereby a greater understanding is given on which forces we can apply and how much velocity is required to e.g get to different positions within the prosthesis' capability.\\
Testing and fine tuning was a large part of the implementation. After several iterations of testing let to this final product, the first implementation let the project team rethink their options due to the lag experienced when using MATLAB. Fortunately, this leads to making a more standalone version the was not tied to a computer to operate.




%designing a system using sEMG sensors and an accelerometer to control the crustrawler. This was done by designing the neccesary software needed and assembling the needed hardware systems. The final solution uses an accelerometer to change motor choice and the EMG input to rotate around the chosen joint. this allows for easy control with a minimal amount of non userfriendly variables such as running in to singularities.

%\textit{the choice of controlling the manipulator in joint space have proven to increase the user friendliness of the system while minimizing the potential troubles that could occur if controlling the end effecter in cartesean space.}
\chapter{Future work}
This report presents a proof of concept, however in order to refine the product, more time, money and work would need to be consumed. 
If future work were to be made, then this prof of concept could be refined into a marketable option.
The project group have several notes on potential future work, that would improve the product.
\paragraph{Users} 
To improve the users experience with the manipulator controls system, it would be advisable to interview potential users, and use their feedback to improve the system. It would also be useful to station the system long term with a user, and user the feedback to refine the product.
\paragraph{Gray Box/ sensor box} 
This device is attached to the user and as such would require some attention. 
The model of the device could be improved to heighten the users experience. 
Some point to be considered would be:
\begin{itemize}
    \item to model an easy way to attache the device to the users and do so ergonomically. 
    \item To improve the electrode attachment method, to ease the attachment, detachment and relocation, done by the user.
    \item As the use of the same muscles would cause strain and fatigue over time, a device could be designed to ease the relocation of the sensors while using several on board preset muscle thresholds to be easily chosen depending on the electrode placement. 
\end{itemize}

\paragraph{Manipulator}
The crustcrawler have been very usefull in developing, the product, however in the future other options should be considered, as there might some that would improve the user experience by providing more Degrees of freedom and as such more freedom of movement for the operator.
another way to improve the product would be to develop improved mounting options for the user. this could be on a wheelchair, on a tabletop or with the focus on easily being able to adjust its position in relation to the user.
The design aesthetics of the body of the manipulator could be refined to better fit into society and to minimise accidental tampering with the hardware eg. pulling out cables and such. thus modelling a shell around the hardware on the base.
\paragraph{use cases}
    After working with this solution it has become clear that it has the potential to be marketed to a wide audience. while it has been developed with the goal of helping people with shoulder articulation it can also be used in many situations where theres a need to remotely manipulate the environment, though several refinements should be made depending on the use case.
\paragraph{Dangers}
If this product were to be used by humans, then the system would require a more firm review of safety measures, than this report has provided. as such a risk assessment should be made to identify potential hazards in using this solution and its effect on the surrounding world.
\paragraph{Requirements not met:}

