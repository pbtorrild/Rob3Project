\begin{lstlisting}[frame=single,language=Arduino]
int a = 2;  //For displaying segment "a" on 7 segment sisplay
int b = 3;  //Segment "b"
int c = 4;  //Segment "c"
int d = 5;  //Segment "d"
int e = 6;  //Segment "e"
int f = 7;  //Segment "f"
int g = 8;  //Segment "g"
int h = 9;  //Segment h
int LedMoter1 = 10; // LED on motor one
int LedMoter2 = 11; // LED Moter two
int LedMoter3 = 12; // LED Moter three
int LedMoter4 = 13; // LED Moter four
int incomingByte;      // a variable to read incoming serial data into
void setup() {
  // initialize serial communication:
  Serial.begin(115200);
  // Setup of LEDs and 7seg LED:
  pinMode(a, OUTPUT);
  pinMode(b, OUTPUT);
  pinMode(c, OUTPUT);
  pinMode(d, OUTPUT);
  pinMode(e, OUTPUT);
  pinMode(f, OUTPUT);
  pinMode(g, OUTPUT);
  pinMode(h, OUTPUT);
  pinMode(LedMoter1, OUTPUT);
  pinMode(LedMoter2, OUTPUT);
  pinMode(LedMoter3, OUTPUT);
  pinMode(LedMoter4, OUTPUT);
}
void loop() {
  // see if there's incoming serial data:
  if (Serial.available() > 0) {
    incomingByte = Serial.read();
    if (incomingByte == '1') {
      digitalWrite(LedMoter1, HIGH); delay(1);
      digitalWrite(LedMoter2, LOW); delay(1);
      digitalWrite(LedMoter3, LOW); delay(1);
      digitalWrite(LedMoter4, LOW); delay(1);
      digitalWrite(a, LOW); delay(1);
      digitalWrite(b, HIGH); delay(1);
      digitalWrite(c, HIGH); delay(1);
      digitalWrite(d, LOW); delay(1);
      digitalWrite(e, LOW); delay(1);
      digitalWrite(f, LOW); delay(1);
      digitalWrite(g, LOW); delay(1);
      digitalWrite(h, HIGH); delay(1);
    }
    if (incomingByte == '2') { ...
    }
    if (incomingByte == '3') { ...
    }
    if (incomingByte == '4') { ...
    }
  }
  return;
}
\end{lstlisting} \label{fig:AS}
\begin{figure}[H]
    \centering
    \caption{The source code for visualising of MotorSelect}
\end{figure}



