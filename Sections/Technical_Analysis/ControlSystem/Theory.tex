%\paragraph{PD} This controller supplements the P controller with a derivative element. This combination is used when a fast rise time is of importance. The added benefits compared to the p-controller is added system stability. It achieves this by reacting to the system errors over time.
%\paragraph{PID} this controller contains the precision of the PI controller, while supplementing it with the speed of the PD controller. Through calibration it can reach a steady state error of 0 , have a fast rise time, and have a stable system.
%The transfer function of the PID controller is:
%\begin{equation}\label{Trans1}
%D(s) = k_P + \frac{k_I}{s} + k_D \cdot s
%\end{equation}
%*\begin{equation}
    %\frac{r(s)}{e(s)}=\frac{100}{PB}\cdot1+\frac{1}{s}\cdot t_R+\frac{s\cdot t_D}{1+s\cdot alpha\cdot t_D}
%\end{equation}
%In equation \ref{Trans1} kP is the proportional term, kI is the integral term and kD is the derivative term.
