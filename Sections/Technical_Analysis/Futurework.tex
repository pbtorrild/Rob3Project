\section{Future work}
This report presents a proof of concept, however, in order to refine the product, more time, money and work would need to be utilised. 
If future work were to be done, then this proof of concept could be refined into a marketable option.
The project group have several notes on potential future work, that would improve the product.

\paragraph{Users} 
To improve the users' experience with the manipulator control system, it would be advisable to interview potential users and use their feedback to improve the system. It would also be useful to provide a long-term user with the system and use the feedback to refine the product further. It is advisable to postpone the usage of test-users till after the solution is technically refined.

\paragraph{Measuring box} 
The measuring box is placed on the user and as such would require some attention. 
The casing of the device could be improved to heighten the users' comfort and ease of use. 
Some point to be considered would be:
\begin{itemize}
    \item To model an easy way to attach the device to the users and do so ergonomically. 
    \item To improve the electrode attachment method, to ease the attachment, detachment, placement and relocation done by the user.
    \item As the use of the same muscles could cause strain and fatigue over time, a device could be designed to ease the relocation of the sensors while using several onboard preset muscle thresholds to be easily chosen depending on the chosen electrode placement. 
\end{itemize}

\paragraph{Manipulator}
The CrustCrawler have been very useful in developing the product, however in the future other options should be considered, as there might be some that could improve the user-experience by providing more degrees of freedom and as such more freedom of movement for the operator, though this would also increase the complexity of the control method for the user.\\
Another way to improve the product would be to develop improved mounting options for the user. This could be on a wheelchair or on a tabletop.\\
The aesthetics of the manipulator could be refined to better fit into society and to minimise accidental tampering with the hardware e.g. accidentally pulling out cables. Thus modelling a shell around the hardware on the base would be advised.\\
\paragraph{use cases}
    After working with this solution, it has become clear that it has the potential to be marketed to a wider audience. While it has been developed with the goal of helping people with shoulder disarticulation, it can also be used in many other situations where there is a need to manipulate the environment remotely.
\paragraph{Dangers}
If this product were to be brought to the public, then the system would require a more firm review of safety measures, than this report has provided. As such a risk assessment should be made to identify potential hazards in using this solution and its effect on the surrounding world before release.


