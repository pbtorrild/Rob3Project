\chapter{Future work}
This report presents a proof of concept, however in order to refine the product, more time, money and work would need to be consumed. 
If future work were to be made, then this prof of concept could be refined into a marketable option.
The project group have several notes on potential future work, that would improve the product.
\paragraph{Users} 
To improve the users experience with the manipulator controls system, it would be advisable to interview potential users, and use their feedback to improve the system. It would also be useful to station the system long term with a user, and user the feedback to refine the product.
\paragraph{Gray Box/ sensor box} 
This device is attached to the user and as such would require some attention. 
The model of the device could be improved to heighten the users experience. 
Some point to be considered would be:
\begin{itemize}
    \item to model an easy way to attache the device to the users and do so ergonomically. 
    \item To improve the electrode attachment method, to ease the attachment, detachment and relocation, done by the user.
    \item As the use of the same muscles would cause strain and fatigue over time, a device could be designed to ease the relocation of the sensors while using several on board preset muscle thresholds to be easily chosen depending on the electrode placement. 
\end{itemize}

\paragraph{Manipulator}
The crustcrawler have been very usefull in developing, the product, however in the future other options should be considered, as there might some that would improve the user experience by providing more Degrees of freedom and as such more freedom of movement for the operator.
another way to improve the product would be to develop improved mounting options for the user. this could be on a wheelchair, on a tabletop or with the focus on easily being able to adjust its position in relation to the user.
The design aesthetics of the body of the manipulator could be refined to better fit into society and to minimise accidental tampering with the hardware eg. pulling out cables and such. thus modelling a shell around the hardware on the base.
\paragraph{use cases}
    After working with this solution it has become clear that it has the potential to be marketed to a wide audience. while it has been developed with the goal of helping people with shoulder articulation it can also be used in many situations where theres a need to remotely manipulate the environment, though several refinements should be made depending on the use case.
\paragraph{Dangers}
If this product were to be used by humans, then the system would require a more firm review of safety measures, than this report has provided. as such a risk assessment should be made to identify potential hazards in using this solution and its effect on the surrounding world.
\paragraph{Requirements not met:}

