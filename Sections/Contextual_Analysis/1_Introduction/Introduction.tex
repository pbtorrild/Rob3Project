\chapter{Introduction}\label{ch:Introduction}
In Denmark today, some people need assistance in their daily life due to their disabilities. This report seeks to assist some of these people by increasing their options in interacting with their surroundings.\\

This report will investigate the problems involved in bettering the lives of people suffering from loss of limbs, with a specific focus on users with shoulder disarticulation and how a robotic manipulator could assist them.\\

An investigation into the probability of using a manipulator such as a CrustCrawler in assisting people suffering from lack of limbs, in their lives, and will create a proof of concept showcasing this.

\section{Initial Problem Formulation}
The initial problem formulation serves a guideline to the research and contextual analysis and ensures that the problems the report seeks out to solve, is described in a satisfactory manner. This is done by creating a short series of questions, to answer in the contextual analysis. These questions are seen below.\\ 
\noindent \textbf{\textit{How can a person with a shoulder disarticulation (upper limb amputees) physically manipulate the world, with use of a control system design implemented in a robotic manipulator?}}
\begin{itemize}
    \item What kind of robotic prosthesis would be suitable to assist a person with a shoulder disarticulation?
    \item What control method would be relevant for a user suffering from shoulder disarticulation?
\end{itemize}
