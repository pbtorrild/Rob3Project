\chapter{Introduction}\label{ch:Introduction}

\info{This section will change during the project}
    
In today's world, people want to live to the best of their ability. People try to improve their lives with material goods and continue to seek more and better devices. Some of these devices provide a higher quality of life than others and becomes a part of the user.\\
This report will investigate the problems involved in bettering the lives of people suffering from loss of limbs, with a specific focus on shoulder disarticulation along with the implementation of robotic control systems in the prostheses, and how these can help in the daily life, while using a minimal amount of muscles to manipulate a robotic arm to do a given task.\\

\section{Initial Problem Formulation}
The initial problem formulation serves a guideline to the research and contextual analysis, and insures that the problems the report seeks out to solve, is described in a satisfactory manner. This is done by creating a Short series a questions, to answer in the contextual analysis. These questions are seen below.\\

\noindent How can a person with a shoulder disarticulation (upper limb amputees) physically manipulate the world, with use of a robotic control system design implemented in a prostheses?
\begin{itemize}
    \item What kind of robotic prosthesis would be suitable to assist a person with a shoulder disarticulation?
    \item What control method would be relevant for a user suffering from shoulder disarticulation?
\end{itemize}
