\chapter{Final problem formulation and requirements}\label{ch:FinalProb}
\section{Conclusion on previous chapters}
People with shoulder disarticulation would benefit from a prosthesis to assist them in their daily life. A prosthetic system could consists of several types. These options have been investigated and described. The rest of this report will focus on the use of a electric prosthesis using a control system. When choosing said control system several viable options are available, but the most suitable ones with ease of use for the user would be using sEMG's sensors as the control input.\\
\section{Final problem formulation}
%%%%%%%%%%%%%%%%%%%%
The user relies on technology to get general tasks done without help from the outside world.
Relying on robotic technology and the usage of control systems these tasks will be normalised and usable to the end-user.\\
\noindent
%%%%%%%%%%%%%%%%%%%% 
\textbf{\textit{\centering How can a person with a shoulder disarticulation through the use of biological signals control a robotic manipulator?}}\\

To have a better understanding of how to implement such a control method on a robotic manipulator, a series of questions have to be answered. 
\begin{itemize}
    \item How to analyse muscular contraction and implement it in the control of the manipulator?
    \item How to implement control systems that is accurate and can handle disturbance input?
    \item How can the manipulator be analysed through a mathematical model?
    \item How to safely connect the user to the manipulator?
\end{itemize}


\section{User requirements}\label{Requirements}
Requirements for the users is needed to understand what the technical requirements have to include. E.g. when a person with a limb loss wants to drink from a cup, the technicalities have to ensure that the arm moves up to the users' mouth with the cup.\\


\begin{table}[H]
    \centering  
\begin{tabular}{ |P{1.5cm}||P{4.5cm}||P{8.5cm}|}
 \hline
 \multicolumn{3}{|c|}{\textbf{User Requirements}} \\
 \hline
 Req. ID & Name & Requirement  \\
 \hline
 \userreqid{Drink} & Basic functions of a robot arm & Drink a glass of water through a straw.  \\
 \hline
 \userreqid{Input device} & Input device & Input device has to cause the least amount of difficulties.  \\
 \hline
 \userreqid{Safety} & Safety & Has to be safe for the user to use.  \\
 \hline
\end{tabular}
\caption{Table of User requirements}
    \label{tab:UReq}
\end{table}

\section*{Technical requirements}\label{T-Requirements}
The user requirements will form the bedrock of the technical requirements. The technical requirements serves as a guideline for this project.
\begin{table}[H]
    \centering  
\begin{tabular}{ |P{1.5cm}||P{11.5cm}||P{1.5cm}|  }
 \hline
 \multicolumn{3}{|c|}{\textbf{Technical Requirements}} \\
 \hline
 Req. ID & Requirement & Parent \\
 \hline
  \reqid \label{test:EMG}  & Thresholds to define different types of movement is required. & U\ref{ureq:Drink} \\
 \hline
 \reqid \label{req:extension}  &  Lift a  minimum 0.5kg & U\ref{ureq:Drink}  \\
 \hline
 %\reqid \label{req:extension2}&End effector carry a minimum of 0.5kg  & U\ref{ureq:Drink}  \\
 %\hline
 %\reqid \label{req:EMGfatigue} & sEMG to prevent muscle fatigue &U\ref{ureq:Input device}\\
 %\hline
 \reqid  \label{req:imu} &Use inertial measuring unit to select motor &U\ref{ureq:Input device}\\
 \hline
 
 \reqid \label{req:electrocution}& EMG system Should be self-contained to ensure no risk of electrocution with the use of wireless connection to manipulator. & U\ref{ureq:Safety}\\
 \hline
 \reqid \label{test:Latency}  & Maximum of latency 1.0 seconds &U\ref{ureq:Input device}\\
 \hline 
 \reqid \label{req:force} & May produce a maximum force of 194.50 N\cite{force} &  U\ref{ureq:Safety}  \\
 \hline
\end{tabular}
\caption{Table of Technical requirements}
    \label{tab:TReq}
\end{table}
\\


%\section{Delimitation's}\label{Delimitations}
%To design a complete prosthesis for a person with disarticulation requires that there are money and time available to do so. Due to the lack of money and time in this project, a concept based prosthesis will be built upon the design of the CrustCrawler robotic manipulator from CrustCrawler robotics, this utilises the servos from Dynamixel. \\
%The controls for the prosthesis will also be limited to two sEMG signals and three signals from an accelerometer, the number of signals from the user will limit the control design approach. All these components will be described in the next chapters.\\
%In order to save time, this project will not create a shoulder mounting and the placement of the robotic prosthesis is placed on a table, where the control system can still be created and tested.



